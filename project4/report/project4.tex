\documentclass[a4paper, fontsize=11pt]{article}

\usepackage{amsmath,amsfonts,amsthm} % Math packages
\usepackage[english]{babel} % English language/hyphenation
%\usepackage{hyperref}
\usepackage{listings}
\usepackage{color}
\usepackage{verbatim}
\usepackage{graphicx}
\usepackage{float}
\usepackage{diagbox}
\usepackage{multirow}
\usepackage{subcaption}

\usepackage{pgfplotstable, booktabs, mathpazo}

\pgfplotstableset{
    every head row/.style={before row=\toprule,after row=\midrule},
    every last row/.style={after row=\bottomrule},
%    alias/{$T$}/.initial=0,
%    alias/{$E$}/.initial=1,
%    alias/{$C_V$}/.initial=2,
%    alias/{$M$}/.initial=3,
%    alias/{$chi$}/.initial=4,
%    alias/{$abs(M)$}/.initial=5,
}



\usepackage[colorlinks=true,linkcolor=black,urlcolor=blue,
citecolor=blue]{hyperref}

%\bibliographystyle{ieeetr}
\bibliographystyle{apalike}

\definecolor{dkgreen}{rgb}{0,0.6,0}
\definecolor{gray}{rgb}{0.5,0.5,0.5}
\definecolor{mauve}{rgb}{0.58,0,0.82}


\lstset{frame=tb,
  language=C++,
  aboveskip=3mm,
  belowskip=3mm,
  showstringspaces=false,
  columns=flexible,
  basicstyle={\small\ttfamily},
  numbers=none,
  numberstyle=\tiny\color{gray},
  keywordstyle=\color{blue},
  commentstyle=\color{dkgreen},
  stringstyle=\color{mauve},
  breaklines=true,
  breakatwhitespace=true,
  tabsize=3
}


\begin{document}

\title{Project 4 FYS4150 \\ Phase transitions in magnetic systems}
\author{Marius Holm}

%----------------------------------------------------------------------------------------
%	PROBLEM 1
%----------------------------------------------------------------------------------------
\maketitle


\section{Abstract}
We give a brief introduction to the Ising model in 1 and 2 dimensions, as well as Monte Carlo integration. We derive the analytic equations for the expected mean energy, mean magnetization, specific heat and magnetic susceptibility, for the simple case of a system of $2 \times 2$ $(L=2)$ spins. We compare the analytic result with our numerical solutions. 

\paragraph{}
We study the number of Monte Carlo cycles needed in order to reach equilibrium states, as well as how the number of Monte Carlo cycles affect the expected mean energy and mean magnetization. We present plots of the expected mean energy per spin, mean magnetization, specific heat, and magnetic susceptibility as functions of $T$, for $L = 40, 60, 80, 100$. The plots indicates that a phase transition happens in the interval $kT/\text{J} \in [2.25, 2.30]$. This fits well with the known theory where $kT_{C}/\text{J} \approx 2.269$.  

\section{Introduction}
In this project we will study the Ising model in two dimensions as a model of phase transitions. At a given critical temperature, this model exhibits a phase transition from a magnetic phase to a phase with zero magnetization. This is a so-called binary system where each object on the lattice can only take two values. In our system we will use spins that either point up or down. We can for example choose to associate spins pointing up with 1, and spins pointing down with 0. 

\paragraph{}
In it's simplest form the energy of the Ising model is expressed as, without an externally applied magnetic field,

\begin{equation}
E = -\text{J} \sum^N_{<kl>} s_{k} s_{l},
\end{equation}

with $s_{k}=\pm 1$. $N$ is the total number of spins in our system and $J$ is a coupling constant expressing the strength of the interaction between neighboring spins. The symbol $<kl>$ indicates that we sum over the nearest neighbors only. For non-neighboring spins the interaction is so weak that we often can ignore this contribution. We also assume that we have ferromagnetic ordering in our system, by assuming $J> 0$. For this project we'll use periodic boundary conditions and the Metropolis algorithm.


\paragraph{}
We give a brief introduction to the Ising model as well as phase transitions near the critical temperature. For a simple system of $2 \times 2$ spins, $L = 2$ we present the analytical expressions for the partition function and the corresponding expectation values for the energy $E$, the mean absolute value of the magnetic moment $\lvert M \rvert$ (we refer to this as the mean magnetization), the specific heat $C_{V}$ and the susceptibility $\chi$ as functions of $T$ using periodic boundary conditions. We then present and discuss the numerical results obtained from our code.



\section{Methods}


\subsection{Ising model}
We won't go deep into the details of the Ising model, but we will give a brief introduction of the basics. For those interested in the specific details of the Ising model and statistical physics we will cite some relevant literature.


\paragraph{}
In statistical physics an ensemble is a very important underlying part for defining thermodynamical quantities. And ensemble is a collection of microphysics systems from which we derive expectation values and thermodynamical properties related to experiments. In our project we base our calculations on the Canonical ensemble, which is one of the most used ensembles. For the Ising model the energy is in its simplest form given as 

\begin{equation}
E = -\text{J} \sum^N_{<kl>} s_{k} s_{l} - \mathcal{B} \sum^N_{k} s_{k},
\end{equation}

where $s_{k}= \pm 1$, $N$ is the total number of spins, $\text{J}$ is the coupling constant and $\mathcal{B}$ is an external magnetic field interacting with the magnetic moment created by the spins. The symbol $<kl>$ indicates that we sum over nearest neighbors only. For our system we will take $\mathcal{B}$ to be zero. We also assume $\text{J}>0$ which is energetically favorable to aligning spins. We will present more details on other important aspects of the Ising model and the canonical ensemble in the results section where we derive analytical results for a system of $2 \times 2$ spins. 

\paragraph{}
A goal of our project is to determine whether a phase transition is taking place. A phase transition is marked by abrupt macroscopic changes as external parameters are changed, such as an increase in temperature. The point at which the phase transition takes place is called a critical point. The Ising model exhibits what is known as a second-order phase transition since the heat capacity diverges. What this actually means is that below a given critical temperature $T_{C}$, the Ising model exhibits spontaneous magnetization with $\langle \mathcal{M} \rangle \neq 0$. Above $T_{C}$ the average magnetization is zero. The mean magnetization approaches zero at $T_{C}$ with an infinite slope. Such behavior is an example of what is called a critical phenomena. \cite{statphys}

\subsection{Monte Carlo method}
The Monte Carlo method is the basis for the code implemented in this project. 



\subsection{Implementation}
The code for this project is heavily based on code found in the GitHub repository \href{https://github.com/CompPhysics/ComputationalPhysics}{GitHub FYS4150} for the course FYS4150 at UiO. We developed our own python code for plotting selected results. The Metropolis algorithm with the use of Monte Carlo cycles is explained in the lecture notes on statistical physics by Hjort-Jensen, on page 12 and onwards. \cite{statphys}


\paragraph{}
All our code, calculations, and plots used can be found in \href{https://github.com/MariusHolm/FYS4150}{my GitHub repository}.

\section{Results}


\subsection{Analytic expressions}
We assume we have a system of two spins in each dimension, denoted as $L = 2$, with periodic boundary conditions. For a square lattice with periodic boundary conditions we have $2^4 = 16$ states. 


\begin{table}[h!]
\centering
\begin{tabular}{|c|| >{\centering}m{1cm}| >{\centering}m{1cm}| >{\centering}m{1cm}| >{\centering}m{1cm}|}
\toprule
\multicolumn{5}{c}{Spin configurations}
%Spin configurations
\\ \hline
\diagbox{$k$}{$j$} & $1$ & $2$ & $3$ & $4$
\tabularnewline
\hline
\multirow{2}{*}{$1$} & $\uparrow \uparrow$ & $\uparrow \uparrow$ & 
$\uparrow \uparrow$ & $\uparrow \uparrow$ 
\tabularnewline
& $\uparrow \uparrow$ & $\uparrow \downarrow$ & 
$\downarrow \uparrow$ & $\downarrow \downarrow$
\tabularnewline
\hline
\multirow{2}{*}{$2$} & $\uparrow \downarrow$ & $\uparrow \downarrow$ & 
$\uparrow \downarrow$ & $\uparrow \downarrow$ 
\tabularnewline
& $\uparrow \uparrow$ & $\uparrow \downarrow$ & 
$\downarrow \uparrow$ & $\downarrow \downarrow$
\tabularnewline
\hline
\multirow{2}{*}{$3$} & $\downarrow \uparrow$ & $\downarrow \uparrow$ & 
$\downarrow \uparrow$ & $\downarrow \uparrow$ 
\tabularnewline
& $\uparrow \uparrow$ & $\uparrow \downarrow$ & 
$\downarrow \uparrow$ & $\downarrow \downarrow$
\tabularnewline
\hline
\multirow{2}{*}{$4$} & $\downarrow \downarrow$ & $\downarrow \downarrow$ & 
$\downarrow \downarrow$ & $\downarrow \downarrow$ 
\tabularnewline
& $\uparrow \uparrow$ & $\uparrow \downarrow$ & 
$\downarrow \uparrow$ & $\downarrow \downarrow$
\tabularnewline \hline
\end{tabular}
\caption{Spin configurations for a $2 \times 2$ system.}
\label{spinconfig}
\end{table}





In order to calculate the expectation values for the mean energy $\langle E \rangle$ and the mean magnetization $\langle \mathcal{M} \rangle$ in statistical physics at a given temperature, we need a probability distribution

\begin{equation}
P_{i}(\beta) = \dfrac{e^{-\beta E_{i}}}{Z}
\end{equation}

which is known as the Boltzmann distribution with $\beta = 1/kT$, where $k$ is the Boltzmann constant, $E_{i}$ the energy of a state $i$, and $Z$ is the partition function for the canonical esemble defined as 

\begin{equation}
Z = \sum^M_{i=1} e^{-\beta E_{i}}
\end{equation}

where the sum extends over all microstates $M$. 

\paragraph{}
With periodic boundary conditions the energy for a given configuration $i$ is given by

\begin{equation}
E_{i} = - \text{J} \sum^N_{<kl>} s_{k} s_{l}. 
\end{equation}

For details on the calculations and more theoretical background please see. \cite{H-Jensen}

\paragraph{}

\begin{table}[h!]
\centering
\begin{tabular}{|c||c|r|r|}
\toprule
\multicolumn{4}{c}{Energy and magnetization}
%Spin configurations
\\ \hline
Spins $\uparrow$ & Degeneracy & $E$ & $\mathcal{M}$
\tabularnewline
\hline
4 & 1 & $-8$ J & $4$
\tabularnewline \hline
3 & 4 & $0$ & $2$
\tabularnewline \hline
2 & 4 & $0$  & $0$
\tabularnewline \hline
2 & 2 & $8$ J & $0$
\tabularnewline \hline
1 & 4 & $0$  & $-2$
\tabularnewline \hline
0 & 1 & $-8$ J & $-4$
\tabularnewline \hline
\end{tabular}
\caption{Energy and magnetization of the different spin configurations described in table \ref{spinconfig}.}
\label{EM}
\end{table}

We can now find an analytic expression for the partition function for our $2 \times 2$ system. We have $M = 16$ different configurations. For the different configurations we have two configurations with $E = -8 \, \text{J}$, two with $E = 8 \, \text{J}$, and $12$ of $E=0$. Thus we have

\begin{equation}
Z = 2 e^{-8\, \text{J} \, \beta} + 2 e^{8 \,\text{J} \, \beta} + 12,
\end{equation}

where the first term is a result from the two configurations $j,k=1,1$ and $j,k = 4,4$. The second term comes from configurations $j,k = 3,2$ and $j,k = 2,3$, while the last term is the contribution from the rest of the configurations with $E=0$.

\paragraph{}
The corresponding expectation values for the energy is given by

\begin{equation}
\langle E \rangle = \sum^M_{i=1} E_{i} \, P_{i}(\beta) = \sum^M_{i=1} E_{i} \, \dfrac{e^{-\beta \, E_{i}}}{Z} = -\dfrac{J}{Z} \left(16\, e^{8 \, \text{J} \, (k \, T)^{-1}} - 16 \, e^{-8 \, \text{J} \, (k \, T)^{-1}} \right)
\end{equation} 
I replace $\beta = (k \, T)^{-1}$ in the results in order to make it clearer to myself and the reader that they are functions of $T$. The variance is defined as

\begin{equation}
\sigma^2_{E} = \left( \langle E^2 \rangle - \langle E \rangle^2 \right) = \dfrac{1}{Z} \sum^M_{i=1} E_{i}^2 \, e^{-\beta \, E_{i}} - \left( \dfrac{1}{Z} \sum^M_{i=1} E_{i} \, e^{-\beta \, E_{i}} \right)^2
\end{equation}

We use this result in order to find the specific heat  $C_{V}$ as

\begin{equation}
C_{V} = \dfrac{1}{k_{B} \, T^2} \left( \langle E^2 \rangle - \langle E \rangle^2 \right)
\end{equation}

\begin{align}
\langle E^2 \rangle &= \dfrac{1}{Z} \sum^M_{i=1} E_{i}^2 \, e^{-\beta \, E_{i}} = \dfrac{1}{Z} \left( 2 \cdot \left( -8 \text{J} \right)^2 \, e^{8 \, \text{J} \, \beta} + 2 \cdot \left( 8 \text{J} \right)^2 \, e^{-8 \, \text{J} \, \beta} \right)
\\
&= \dfrac{\text{J}^2}{Z} \left( 128 \, e^{8 \, \text{J} \, \beta} + 128 \, e^{-8 \, \text{J} \, \beta} \right)
\end{align}


\begin{align}
C_{V} &= \dfrac{1}{k_{B} \, T^2} \left( \dfrac{\text{J}^2}{Z} \left( 128 \, e^{8 \, \text{J} \, \beta} + 128 \, e^{-8 \, \text{J} \, \beta} \right) - \left(-\dfrac{\text{J}}{Z} \left(16 \, e^{8 \, \text{J} \, \beta} - 16 \, e^{-8 \, \text{J} \, \beta} \right) \right)^2 \right)
\\
&= 
\dfrac{1}{k_{B} \, T^2} \left( \dfrac{\text{J}^2}{Z} \left( 128 \, e^{8 \, \text{J} \, (k \, T)^{-1}} + 128 \, e^{-8 \, \text{J} \, (k \, T)^{-1}} \right) 
- 
\dfrac{\text{J}^2}{Z^2} \left(256 \, e^{16 \, \text{J} \, (k \, T)^{-1}} - 512 + 256 \, e^{-16 \, \text{J} \, (k \, T)^{-1}} \right) \right) 
\end{align}



\paragraph{}
The mean absolute magnetization $\langle \mathcal{M} \rangle$ can be calculated as

\begin{equation}
\langle \mathcal{M} \rangle = \sum^M_{i=1} \mathcal{M}_{i} \, P_{i}(\beta) = \dfrac{1}{Z} \sum^M_{i=1} \mathcal{M}_{i} \, e^{-\beta \, E_{i}}
\end{equation}

where $\mathcal{M}_{i}$ is the magnetization for a given configuration $i$ calculated as

\begin{equation}
\mathcal{M}_{i} = \sum^N_{j=1} s_{j}
\end{equation}

where we sum over each spin in the lattice, counting up spins as $+1$ and down spins as $-1$. The magnetization of each "equivalent" configuration is given in table \ref{EM}. As such we can calculate $\langle \mathcal{M} \rangle$ as

\begin{equation}
\langle \mathcal{M} \rangle = \dfrac{1}{Z} \left( 4 \, e^{8 \, \text{J} \, \beta} + 4 (2 \, e^0) + 4 ( -2 \, e^0) -4 \, e^{8 \, \text{J} \, \beta} \right) = 0
\end{equation}

The variance is defined as 

\begin{equation}
\sigma^2_{\mathcal{M}} = \left( \langle \mathcal{M}^2 \rangle - \langle \mathcal{M} \rangle^2 \right) = \dfrac{1}{Z} \sum^M_{i=1} \mathcal{M}_{i}^2 \, e^{-\beta \, E_{i}} - \left( \dfrac{1}{Z} \sum^M_{i=1} \mathcal{M}_{i} \, e^{-\beta \, E_{i}} \right)^2
\end{equation}

\begin{align}
\langle \mathcal{M}^2 \rangle &= \dfrac{1}{Z} \left( 4^2 \, e^{8 \, \text{J} \, \beta} + 4 \, (2^2 \, e^0 ) + 4 \, ((-2)^2 \, e^0) + (-4)^2 \, e^{8 \, \text{J} \, \beta} \right) 
\\
&=\dfrac{1}{Z} \left( 16 \, e^{8 \, \text{J} \, (k \, T)^{-1}} + 32 + 16 \, e^{8 \, \text{J} \, (k \, T)^{-1}} \right) 
\end{align}

We use this result in order to calculate the susceptibility $\chi$ as

\begin{align}
\chi &= \dfrac{1}{k_{B} \, T} \left( \langle \mathcal{M}^2 \rangle - \langle \mathcal{M} \rangle^2 \right)
\\
&=\dfrac{1}{k_{B} \, T} \left( \dfrac{1}{Z} \left( 16 \, e^{8 \, \text{J} \, \beta} + 32 + 16 \, e^{8 \, \text{J} \, \beta} \right)  - 0^2 \right)
\\
&= 
\dfrac{1}{k_{B} \, T \, Z} \left( 16 \, e^{8 \, \text{J} \, (k \, T)^{-1}} + 32 + 16 \, e^{8 \, \text{J} \, (k \, T)^{-1}} \right) 
\end{align}


\cite{H-Jensen}

\subsection{Numerical comparison}

We assume a system of $2 \times 2$ spins with periodic boundaries. We insert the temperature $T = 1$ in units $kT/\text{J}$. Thus we have $\text{J} \beta = \text{J}/(kT) = 1^{-1} = 1$. The analytic result then becomes 

\begin{equation}
\langle E \rangle = \dfrac{\text{J}}{2e^{8} + 2e^{-8} + 12} \left(-16\, e^{8} + 16 \, e^{-8} \right) = \dfrac{8\, \text{J}\left(- e^{8} + e^{-8} \right)}{e^{8} + e^{-8} + 6} \approx -7.983928 \, \text{J}
\end{equation}

The expectation value for the mean energy per spin is then

\begin{equation}
\dfrac{\langle E \rangle}{\text{J}} \approx -1.995982
\end{equation}



\begin{table}[h!tb]
\centering
\pgfplotstabletypeset[sci, precision = 5]{../code/results/mpi/datafiles/Lattice_2_1000_1.000000.dat}
\caption{Numerical results for a system with $L=2$ with 1000 Monte Carlo cycles.}
\end{table}

\begin{table}[h!tb]
\centering
\pgfplotstabletypeset[sci, precision = 5]{../code/src/opt_results/opt_mpi/results_2_10000_0.990000.dat}
\caption{Numerical results for a system with $L=2$ with 10 000 Monte Carlo cycles.}
\end{table}

\begin{table}[h!tb]
\centering
\pgfplotstabletypeset[sci, precision = 5]{../code/src/opt_results/opt_mpi/results_2_100000_0.990000.dat}
\caption{Numerical results for a system with $L=2$ with 100 000 Monte Carlo cycles.}
\end{table}

\begin{table}[h!tb]
\centering
\pgfplotstabletypeset[sci, precision = 5]{../code/src/opt_results/opt_mpi/results_2_1000000_0.990000.dat}
\caption{Numerical results for a system with $L=2$ with 1 000 000 Monte Carlo cycles.}
\end{table}

\begin{table}[h!tb]
\centering
\pgfplotstabletypeset[sci, precision = 5]{../code/src/opt_results/opt_mpi/results_2_10000000_0.990000.dat}
\caption{Numerical results for a system with $L=2$ with 10 000 000 Monte Carlo cycles.}
\end{table}

\begin{table}[h!tb]
\centering
\pgfplotstabletypeset[sci, precision = 5]{../code/src/opt_results/opt_mpi/results_2_100000000_0.990000.dat}
\caption{Numerical results for a system with $L=2$ with 100 000 000 Monte Carlo cycles.}
\end{table}



\subsection{Equilibrium situations}











%\begin{table}[h!tb]
%    \centering
%    \caption{Results using $10^5$, $10^6$, $10^7$, $10^8$ Monte Carlo cycles.}
%    \pgfplotstabletypeset[sci, precision=4]{../code/results/mpi/datafiles/Lattice_2_10000_1.000000.dat}
%	\pgfplotstabletypeset[sci, precision=4]{../code/results/mpi/datafiles/Lattice_2_100000_1.000000.dat}
% 	\pgfplotstabletypeset[sci, precision=4]{../code/results/mpi/datafiles/Lattice_2_1000000_1.000000.dat}
%    \pgfplotstabletypeset[sci, precision=4]{../code/results/mpi/datafiles/Lattice_2_10000000_1.000000.dat}
%\end{table}






\subsection{Numerical studies of phase transitions}

We want to study the behavior of the 2D Ising model close to the critical temperature as a function of the lattice size $L \times L$. We calculate and plot the expectation values for $\langle E \rangle$ and $\langle | \mathcal{M} | \rangle$, the specific heat $C_{V}$, and the susceptibility $\chi$ as functions of $T$ for $L=40$, $L=60$, $L=80$, and $L=100$. We plot for the temperature interval $T \in [2.0, 2.4]$ with a temperature step $\Delta T = 0.05$.


\begin{figure}[H]
	\begin{subfigure}[t]{0.5\textwidth}
		\includegraphics[scale=0.4]{../code/src/opt_results/opt_mpi/plots/prob_e/{Energy_40_100000_2.000000}.png}
	\end{subfigure}
	\begin{subfigure}[t]{0.5\textwidth}
		\includegraphics[scale=0.4]{../code/src/opt_results/opt_mpi/plots/prob_e/{Energy_60_100000_2.000000}.png}
	\end{subfigure}
		\caption{Plots of the expectation value of $\langle E \rangle$ as a function of $T$ for $L = 40, 60$. Calculated with 100 000 Monte Carlo cycles.}
\end{figure}

\begin{figure}[H]
	\begin{subfigure}[t]{0.5\textwidth}
		\includegraphics[scale=0.4]{../code/src/opt_results/opt_mpi/plots/prob_e/{Energy_80_100000_2.000000}.png}
	\end{subfigure}
	\begin{subfigure}[t]{0.5\textwidth}
		\includegraphics[scale=0.4]{../code/src/opt_results/opt_mpi/plots/prob_e/{Energy_100_100000_2.000000}.png}
	\end{subfigure}
	\caption{Plots of the expectation value of $\langle E \rangle$ as a function of $T$ for $L = 80, 100$. Calculated with 100 000 Monte Carlo cycles.}
\end{figure}


\begin{figure}[H]
	\begin{subfigure}[t]{0.5\textwidth}
		\includegraphics[scale=0.4]{../code/src/opt_results/opt_mpi/plots/prob_e/{Abs_Magnetization_40_100000_2.000000}.png}
	\end{subfigure}
	\begin{subfigure}[t]{0.5\textwidth}
		\includegraphics[scale=0.4]{../code/src/opt_results/opt_mpi/plots/prob_e/{Abs_Magnetization_60_100000_2.000000}.png}
	\end{subfigure}
\\
	\begin{subfigure}[t]{0.5\textwidth}
		\includegraphics[scale=0.4]{../code/src/opt_results/opt_mpi/plots/prob_e/{Abs_Magnetization_80_100000_2.000000}.png}
	\end{subfigure}
	\begin{subfigure}[t]{0.5\textwidth}
		\includegraphics[scale=0.4]{../code/src/opt_results/opt_mpi/plots/prob_e/{Abs_Magnetization_100_100000_2.000000}.png}
	\end{subfigure}
		\caption{Plots of the expectation value of $\langle | \mathcal{M} | \rangle$ as a function of $T$ with $L = 40, 60, 80, 100$. Calculated for 100 000 Monte Carlo cycles.}
		\label{fig:absmag}
\end{figure}


\begin{figure}[H]
	\begin{subfigure}[t]{0.5\textwidth}
		\includegraphics[scale=0.4]{../code/src/opt_results/opt_mpi/plots/prob_e/{Specific_Heat_40_100000_2.000000}.png}
	\end{subfigure}
	\begin{subfigure}[t]{0.5\textwidth}
		\includegraphics[scale=0.4]{../code/src/opt_results/opt_mpi/plots/prob_e/{Specific_Heat_60_100000_2.000000}.png}
	\end{subfigure}
\\
	\begin{subfigure}[t]{0.5\textwidth}
		\includegraphics[scale=0.4]{../code/src/opt_results/opt_mpi/plots/prob_e/{Specific_Heat_80_100000_2.000000}.png}
	\end{subfigure}
	\begin{subfigure}[t]{0.5\textwidth}
		\includegraphics[scale=0.4]{../code/src/opt_results/opt_mpi/plots/prob_e/{Specific_Heat_100_100000_2.000000}.png}
	\end{subfigure}
		\caption{Plots of the specific heat as a function of $T$ for $L = 40, 60, 80, 100$. Calculated with 100 000 Monte Carlo cycles.}
		\label{fig:specific}
\end{figure}



\begin{figure}[H]
	\begin{subfigure}[t]{0.5\textwidth}
		\includegraphics[scale=0.4]{../code/src/opt_results/opt_mpi/plots/prob_e/{Susceptibility_40_100000_2.000000}.png}
	\end{subfigure}
	\begin{subfigure}[t]{0.5\textwidth}
		\includegraphics[scale=0.4]{../code/src/opt_results/opt_mpi/plots/prob_e/{Susceptibility_60_100000_2.000000}.png}
	\end{subfigure}
\\
	\begin{subfigure}[t]{0.5\textwidth}
		\includegraphics[scale=0.4]{../code/src/opt_results/opt_mpi/plots/prob_e/{Susceptibility_80_100000_2.000000}.png}
	\end{subfigure}
	\begin{subfigure}[t]{0.5\textwidth}
		\includegraphics[scale=0.4]{../code/src/opt_results/opt_mpi/plots/prob_e/{Susceptibility_100_100000_2.000000}.png}
	\end{subfigure}
	\caption{Plots of the magnetic susceptibility as a function of $T$ for $L = 40, 60, 80, 100$. Calculated with 100 000 Monte Carlo cycles.}
	\label{fig:susceptibility}
\end{figure}

From the plots above we see that there is obviously some drastic changes in our system for $T \in [2.2, 2.4]$. The changes are clearly most visible in the plots of absolute magnetism, specific heat, and the magnetic susceptibility. These first plots were found using data with temperature steps $\Delta T = 0.05$, in order to be more computationally efficient when we looked at the bigger interval of $T \in [2.0, 2.4]$. We now take a closer look at the smaller interval, and we now use a temperature step of $\Delta T = 0.01$.


\subsection{Method stability} 




As we can see from figures \ref{fig:Verlet} and \ref{fig:Euler}, we can achieve pretty nice results using both Euler's forward algorithm as well as the velocity Verlet algorithm. However we notice that in order to get a nice circular orbit we need 100 times more timesteps for Euler's method compared to velocity Verlet. I.e. in order to avoid divergence, Euler's method requires significantly more time steps.


%\paragraph{}
%Testing Anders' code for automatic inclusion of data to table.
%\begin{table}[h!t
%    \centering
%    \caption{The greatest table.}
%    \pgfplotstabletypeset[sci]{error.dat}
%\end{table}

\section{Conclusions}

We found that our numerical method approximates the analytical results for the Ising model for a $2 \times 2$ quite well. The mean energy results are correct to the third decimal for 1000 Monte Carlo cycles. The accuracy improves when we increase the number of cycles, and for $10^8$ cycles we are starting to get very close to the analytical result.



\bibliography{references}
\end{document}